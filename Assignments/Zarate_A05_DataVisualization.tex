% Options for packages loaded elsewhere
\PassOptionsToPackage{unicode}{hyperref}
\PassOptionsToPackage{hyphens}{url}
%
\documentclass[
]{article}
\usepackage{lmodern}
\usepackage{amssymb,amsmath}
\usepackage{ifxetex,ifluatex}
\ifnum 0\ifxetex 1\fi\ifluatex 1\fi=0 % if pdftex
  \usepackage[T1]{fontenc}
  \usepackage[utf8]{inputenc}
  \usepackage{textcomp} % provide euro and other symbols
\else % if luatex or xetex
  \usepackage{unicode-math}
  \defaultfontfeatures{Scale=MatchLowercase}
  \defaultfontfeatures[\rmfamily]{Ligatures=TeX,Scale=1}
\fi
% Use upquote if available, for straight quotes in verbatim environments
\IfFileExists{upquote.sty}{\usepackage{upquote}}{}
\IfFileExists{microtype.sty}{% use microtype if available
  \usepackage[]{microtype}
  \UseMicrotypeSet[protrusion]{basicmath} % disable protrusion for tt fonts
}{}
\makeatletter
\@ifundefined{KOMAClassName}{% if non-KOMA class
  \IfFileExists{parskip.sty}{%
    \usepackage{parskip}
  }{% else
    \setlength{\parindent}{0pt}
    \setlength{\parskip}{6pt plus 2pt minus 1pt}}
}{% if KOMA class
  \KOMAoptions{parskip=half}}
\makeatother
\usepackage{xcolor}
\IfFileExists{xurl.sty}{\usepackage{xurl}}{} % add URL line breaks if available
\IfFileExists{bookmark.sty}{\usepackage{bookmark}}{\usepackage{hyperref}}
\hypersetup{
  pdftitle={Assignment 5: Data Visualization},
  pdfauthor={Camila Zarate Ospina},
  hidelinks,
  pdfcreator={LaTeX via pandoc}}
\urlstyle{same} % disable monospaced font for URLs
\usepackage[margin=2.54cm]{geometry}
\usepackage{color}
\usepackage{fancyvrb}
\newcommand{\VerbBar}{|}
\newcommand{\VERB}{\Verb[commandchars=\\\{\}]}
\DefineVerbatimEnvironment{Highlighting}{Verbatim}{commandchars=\\\{\}}
% Add ',fontsize=\small' for more characters per line
\usepackage{framed}
\definecolor{shadecolor}{RGB}{248,248,248}
\newenvironment{Shaded}{\begin{snugshade}}{\end{snugshade}}
\newcommand{\AlertTok}[1]{\textcolor[rgb]{0.94,0.16,0.16}{#1}}
\newcommand{\AnnotationTok}[1]{\textcolor[rgb]{0.56,0.35,0.01}{\textbf{\textit{#1}}}}
\newcommand{\AttributeTok}[1]{\textcolor[rgb]{0.77,0.63,0.00}{#1}}
\newcommand{\BaseNTok}[1]{\textcolor[rgb]{0.00,0.00,0.81}{#1}}
\newcommand{\BuiltInTok}[1]{#1}
\newcommand{\CharTok}[1]{\textcolor[rgb]{0.31,0.60,0.02}{#1}}
\newcommand{\CommentTok}[1]{\textcolor[rgb]{0.56,0.35,0.01}{\textit{#1}}}
\newcommand{\CommentVarTok}[1]{\textcolor[rgb]{0.56,0.35,0.01}{\textbf{\textit{#1}}}}
\newcommand{\ConstantTok}[1]{\textcolor[rgb]{0.00,0.00,0.00}{#1}}
\newcommand{\ControlFlowTok}[1]{\textcolor[rgb]{0.13,0.29,0.53}{\textbf{#1}}}
\newcommand{\DataTypeTok}[1]{\textcolor[rgb]{0.13,0.29,0.53}{#1}}
\newcommand{\DecValTok}[1]{\textcolor[rgb]{0.00,0.00,0.81}{#1}}
\newcommand{\DocumentationTok}[1]{\textcolor[rgb]{0.56,0.35,0.01}{\textbf{\textit{#1}}}}
\newcommand{\ErrorTok}[1]{\textcolor[rgb]{0.64,0.00,0.00}{\textbf{#1}}}
\newcommand{\ExtensionTok}[1]{#1}
\newcommand{\FloatTok}[1]{\textcolor[rgb]{0.00,0.00,0.81}{#1}}
\newcommand{\FunctionTok}[1]{\textcolor[rgb]{0.00,0.00,0.00}{#1}}
\newcommand{\ImportTok}[1]{#1}
\newcommand{\InformationTok}[1]{\textcolor[rgb]{0.56,0.35,0.01}{\textbf{\textit{#1}}}}
\newcommand{\KeywordTok}[1]{\textcolor[rgb]{0.13,0.29,0.53}{\textbf{#1}}}
\newcommand{\NormalTok}[1]{#1}
\newcommand{\OperatorTok}[1]{\textcolor[rgb]{0.81,0.36,0.00}{\textbf{#1}}}
\newcommand{\OtherTok}[1]{\textcolor[rgb]{0.56,0.35,0.01}{#1}}
\newcommand{\PreprocessorTok}[1]{\textcolor[rgb]{0.56,0.35,0.01}{\textit{#1}}}
\newcommand{\RegionMarkerTok}[1]{#1}
\newcommand{\SpecialCharTok}[1]{\textcolor[rgb]{0.00,0.00,0.00}{#1}}
\newcommand{\SpecialStringTok}[1]{\textcolor[rgb]{0.31,0.60,0.02}{#1}}
\newcommand{\StringTok}[1]{\textcolor[rgb]{0.31,0.60,0.02}{#1}}
\newcommand{\VariableTok}[1]{\textcolor[rgb]{0.00,0.00,0.00}{#1}}
\newcommand{\VerbatimStringTok}[1]{\textcolor[rgb]{0.31,0.60,0.02}{#1}}
\newcommand{\WarningTok}[1]{\textcolor[rgb]{0.56,0.35,0.01}{\textbf{\textit{#1}}}}
\usepackage{graphicx,grffile}
\makeatletter
\def\maxwidth{\ifdim\Gin@nat@width>\linewidth\linewidth\else\Gin@nat@width\fi}
\def\maxheight{\ifdim\Gin@nat@height>\textheight\textheight\else\Gin@nat@height\fi}
\makeatother
% Scale images if necessary, so that they will not overflow the page
% margins by default, and it is still possible to overwrite the defaults
% using explicit options in \includegraphics[width, height, ...]{}
\setkeys{Gin}{width=\maxwidth,height=\maxheight,keepaspectratio}
% Set default figure placement to htbp
\makeatletter
\def\fps@figure{htbp}
\makeatother
\setlength{\emergencystretch}{3em} % prevent overfull lines
\providecommand{\tightlist}{%
  \setlength{\itemsep}{0pt}\setlength{\parskip}{0pt}}
\setcounter{secnumdepth}{-\maxdimen} % remove section numbering

\title{Assignment 5: Data Visualization}
\author{Camila Zarate Ospina}
\date{}

\begin{document}
\maketitle

\hypertarget{overview}{%
\subsection{OVERVIEW}\label{overview}}

This exercise accompanies the lessons in Environmental Data Analytics on
Data Visualization

\hypertarget{directions}{%
\subsection{Directions}\label{directions}}

\begin{enumerate}
\def\labelenumi{\arabic{enumi}.}
\tightlist
\item
  Change ``Student Name'' on line 3 (above) with your name.
\item
  Work through the steps, \textbf{creating code and output} that fulfill
  each instruction.
\item
  Be sure to \textbf{answer the questions} in this assignment document.
\item
  When you have completed the assignment, \textbf{Knit} the text and
  code into a single PDF file.
\item
  After Knitting, submit the completed exercise (PDF file) to the
  dropbox in Sakai. Add your last name into the file name (e.g.,
  ``Fay\_A05\_DataVisualization.Rmd'') prior to submission.
\end{enumerate}

The completed exercise is due on Tuesday, February 23 at 11:59 pm.

\hypertarget{set-up-your-session}{%
\subsection{Set up your session}\label{set-up-your-session}}

\begin{enumerate}
\def\labelenumi{\arabic{enumi}.}
\item
  Set up your session. Verify your working directory and load the
  tidyverse and cowplot packages. Upload the NTL-LTER processed data
  files for nutrients and chemistry/physics for Peter and Paul Lakes
  (both the tidy
  {[}\texttt{NTL-LTER\_Lake\_Chemistry\_Nutrients\_PeterPaul\_Processed.csv}{]}
  and the gathered
  {[}\texttt{NTL-LTER\_Lake\_Nutrients\_PeterPaulGathered\_Processed.csv}{]}
  versions) and the processed data file for the Niwot Ridge litter
  dataset.
\item
  Make sure R is reading dates as date format; if not change the format
  to date.
\end{enumerate}

\begin{Shaded}
\begin{Highlighting}[]
\CommentTok{#1 Set up}
\CommentTok{# Verify working directory and load packages}
\KeywordTok{getwd}\NormalTok{()}
\end{Highlighting}
\end{Shaded}

\begin{verbatim}
## [1] "/Users/camilazarate/OneDrive - Duke University/2 Second semester/Data analytics/Environmental_Data_Analytics_2021"
\end{verbatim}

\begin{Shaded}
\begin{Highlighting}[]
\KeywordTok{library}\NormalTok{(tidyverse)}
\end{Highlighting}
\end{Shaded}

\begin{verbatim}
## -- Attaching packages ------------------------------------------------ tidyverse 1.3.0 --
\end{verbatim}

\begin{verbatim}
## v ggplot2 3.3.2     v purrr   0.3.4
## v tibble  3.0.3     v dplyr   1.0.2
## v tidyr   1.1.2     v stringr 1.4.0
## v readr   1.3.1     v forcats 0.5.0
\end{verbatim}

\begin{verbatim}
## -- Conflicts --------------------------------------------------- tidyverse_conflicts() --
## x dplyr::filter() masks stats::filter()
## x dplyr::lag()    masks stats::lag()
\end{verbatim}

\begin{Shaded}
\begin{Highlighting}[]
\KeywordTok{library}\NormalTok{(cowplot)}

\CommentTok{# Load data}
\NormalTok{PeterPaul <-}\StringTok{ }\KeywordTok{read.csv}\NormalTok{(}\StringTok{"./Data/Processed/NTL-LTER_Lake_Chemistry_Nutrients_PeterPaul_Processed.csv"}\NormalTok{, }\DataTypeTok{stringsAsFactors =} \OtherTok{TRUE}\NormalTok{)}
\NormalTok{PeterPaul.gathered <-}\StringTok{ }\KeywordTok{read.csv}\NormalTok{(}\StringTok{"./Data/Processed/NTL-LTER_Lake_Nutrients_PeterPaulGathered_Processed.csv"}\NormalTok{, }\DataTypeTok{stringsAsFactors =} \OtherTok{TRUE}\NormalTok{)}
\NormalTok{Litter <-}\StringTok{ }\KeywordTok{read.csv}\NormalTok{(}\StringTok{"./Data/Processed/NEON_NIWO_Litter_mass_trap_Processed.csv"}\NormalTok{,}
                   \DataTypeTok{stringsAsFactors =} \OtherTok{TRUE}\NormalTok{)}

\CommentTok{# Change dates }
\KeywordTok{class}\NormalTok{(PeterPaul}\OperatorTok{$}\NormalTok{sampledate)}
\end{Highlighting}
\end{Shaded}

\begin{verbatim}
## [1] "factor"
\end{verbatim}

\begin{Shaded}
\begin{Highlighting}[]
\NormalTok{PeterPaul}\OperatorTok{$}\NormalTok{sampledate <-}\StringTok{ }\KeywordTok{as.Date}\NormalTok{(PeterPaul}\OperatorTok{$}\NormalTok{sampledate, }\DataTypeTok{format =} \StringTok{"%Y-%m-%d"}\NormalTok{) }

\KeywordTok{class}\NormalTok{(PeterPaul.gathered}\OperatorTok{$}\NormalTok{sampledate)}
\end{Highlighting}
\end{Shaded}

\begin{verbatim}
## [1] "factor"
\end{verbatim}

\begin{Shaded}
\begin{Highlighting}[]
\NormalTok{PeterPaul.gathered}\OperatorTok{$}\NormalTok{sampledate <-}\StringTok{ }\KeywordTok{as.Date}\NormalTok{(PeterPaul.gathered}\OperatorTok{$}\NormalTok{sampledate, }\DataTypeTok{format =} \StringTok{"%Y-%m-%d"}\NormalTok{) }

\KeywordTok{class}\NormalTok{(Litter}\OperatorTok{$}\NormalTok{collectDate)}
\end{Highlighting}
\end{Shaded}

\begin{verbatim}
## [1] "factor"
\end{verbatim}

\begin{Shaded}
\begin{Highlighting}[]
\NormalTok{Litter}\OperatorTok{$}\NormalTok{collectDate <-}\StringTok{ }\KeywordTok{as.Date}\NormalTok{(Litter}\OperatorTok{$}\NormalTok{collectDate, }\DataTypeTok{format =} \StringTok{"%Y-%m-%d"}\NormalTok{)}
\end{Highlighting}
\end{Shaded}

\hypertarget{define-your-theme}{%
\subsection{Define your theme}\label{define-your-theme}}

\begin{enumerate}
\def\labelenumi{\arabic{enumi}.}
\setcounter{enumi}{2}
\tightlist
\item
  Build a theme and set it as your default theme.
\end{enumerate}

\begin{Shaded}
\begin{Highlighting}[]
\NormalTok{my.theme <-}\StringTok{ }\KeywordTok{theme_grey}\NormalTok{(}\DataTypeTok{base_size =} \DecValTok{12}\NormalTok{) }\OperatorTok{+}
\StringTok{  }\KeywordTok{theme}\NormalTok{(}\DataTypeTok{axis.text =} \KeywordTok{element_text}\NormalTok{(}\DataTypeTok{color =} \StringTok{"black"}\NormalTok{), }
        \DataTypeTok{legend.position =} \StringTok{"right"}\NormalTok{)}
\KeywordTok{theme_set}\NormalTok{(my.theme)}
\end{Highlighting}
\end{Shaded}

\hypertarget{create-graphs}{%
\subsection{Create graphs}\label{create-graphs}}

For numbers 4-7, create ggplot graphs and adjust aesthetics to follow
best practices for data visualization. Ensure your theme, color
palettes, axes, and additional aesthetics are edited accordingly.

\begin{enumerate}
\def\labelenumi{\arabic{enumi}.}
\setcounter{enumi}{3}
\tightlist
\item
  {[}NTL-LTER{]} Plot total phosphorus (\texttt{tp\_ug}) by phosphate
  (\texttt{po4}), with separate aesthetics for Peter and Paul lakes. Add
  a line of best fit and color it black. Adjust your axes to hide
  extreme values.
\end{enumerate}

\begin{Shaded}
\begin{Highlighting}[]
\NormalTok{tp_by_po4 <-}\StringTok{ }\KeywordTok{ggplot}\NormalTok{(PeterPaul, }\KeywordTok{aes}\NormalTok{(}\DataTypeTok{x =}\NormalTok{ tp_ug, }\DataTypeTok{y =}\NormalTok{ po4, }\DataTypeTok{color =}\NormalTok{ lakename),}
                    \DataTypeTok{shape =}\NormalTok{ lakename) }\OperatorTok{+}
\StringTok{  }\KeywordTok{geom_point}\NormalTok{(}\DataTypeTok{alpha =} \FloatTok{0.8}\NormalTok{, }\DataTypeTok{size =} \FloatTok{1.5}\NormalTok{) }\OperatorTok{+}\StringTok{ }
\StringTok{  }\KeywordTok{labs}\NormalTok{(}\DataTypeTok{title =} \StringTok{"Total phosphorus by phosphate"}\NormalTok{, }\DataTypeTok{color =} \StringTok{"Lake name"}\NormalTok{,}
       \DataTypeTok{y =} \StringTok{"Phosphate (ug/L)"}\NormalTok{, }\DataTypeTok{x =} \StringTok{"Total phosphorus (ug/L)"}\NormalTok{) }\OperatorTok{+}
\StringTok{  }\KeywordTok{ylim}\NormalTok{(}\DecValTok{0}\NormalTok{, }\DecValTok{50}\NormalTok{) }\OperatorTok{+}
\StringTok{  }\KeywordTok{geom_smooth}\NormalTok{(}\DataTypeTok{method =}\NormalTok{ lm, }\DataTypeTok{color =} \StringTok{"black"}\NormalTok{)}
\KeywordTok{print}\NormalTok{(tp_by_po4) }
\end{Highlighting}
\end{Shaded}

\begin{verbatim}
## `geom_smooth()` using formula 'y ~ x'
\end{verbatim}

\includegraphics{Zarate_A05_DataVisualization_files/figure-latex/unnamed-chunk-3-1.pdf}

\begin{enumerate}
\def\labelenumi{\arabic{enumi}.}
\setcounter{enumi}{4}
\tightlist
\item
  {[}NTL-LTER{]} Make three separate boxplots of (a) temperature, (b)
  TP, and (c) TN, with month as the x axis and lake as a color
  aesthetic. Then, create a cowplot that combines the three graphs. Make
  sure that only one legend is present and that graph axes are aligned.
\end{enumerate}

\begin{Shaded}
\begin{Highlighting}[]
\CommentTok{# Change month to factor so it spaces out the months per set of boxplots.}
\NormalTok{PeterPaul}\OperatorTok{$}\NormalTok{month <-}\StringTok{ }\KeywordTok{as.factor}\NormalTok{(PeterPaul}\OperatorTok{$}\NormalTok{month)}

\CommentTok{# Temperature }
\CommentTok{# Fill inside aes to show the legend.  }
\NormalTok{boxplot.temp <-}\StringTok{ }\KeywordTok{ggplot}\NormalTok{(PeterPaul, }\KeywordTok{aes}\NormalTok{(}\DataTypeTok{x =}\NormalTok{ month, }\DataTypeTok{y =}\NormalTok{ temperature_C,}
                                   \DataTypeTok{fill =}\NormalTok{ lakename)) }\OperatorTok{+}
\StringTok{  }\KeywordTok{geom_boxplot}\NormalTok{() }\OperatorTok{+}
\StringTok{  }\KeywordTok{labs}\NormalTok{ (}\DataTypeTok{title =} \StringTok{"Temperature by month"}\NormalTok{, }\DataTypeTok{fill =} \StringTok{"Lake name"}\NormalTok{,}
        \DataTypeTok{y =} \StringTok{"Temperature (C)"}\NormalTok{, }\DataTypeTok{x =} \StringTok{"Month"}\NormalTok{) }
  \KeywordTok{scale_fill_brewer}\NormalTok{(}\DataTypeTok{palette =} \StringTok{"Dark2"}\NormalTok{) }
\end{Highlighting}
\end{Shaded}

\begin{verbatim}
## <ggproto object: Class ScaleDiscrete, Scale, gg>
##     aesthetics: fill
##     axis_order: function
##     break_info: function
##     break_positions: function
##     breaks: waiver
##     call: call
##     clone: function
##     dimension: function
##     drop: TRUE
##     expand: waiver
##     get_breaks: function
##     get_breaks_minor: function
##     get_labels: function
##     get_limits: function
##     guide: legend
##     is_discrete: function
##     is_empty: function
##     labels: waiver
##     limits: NULL
##     make_sec_title: function
##     make_title: function
##     map: function
##     map_df: function
##     n.breaks.cache: NULL
##     na.translate: TRUE
##     na.value: NA
##     name: waiver
##     palette: function
##     palette.cache: NULL
##     position: left
##     range: <ggproto object: Class RangeDiscrete, Range, gg>
##         range: NULL
##         reset: function
##         train: function
##         super:  <ggproto object: Class RangeDiscrete, Range, gg>
##     rescale: function
##     reset: function
##     scale_name: brewer
##     train: function
##     train_df: function
##     transform: function
##     transform_df: function
##     super:  <ggproto object: Class ScaleDiscrete, Scale, gg>
\end{verbatim}

\begin{Shaded}
\begin{Highlighting}[]
\KeywordTok{print}\NormalTok{(boxplot.temp) }
\end{Highlighting}
\end{Shaded}

\includegraphics{Zarate_A05_DataVisualization_files/figure-latex/unnamed-chunk-4-1.pdf}

\begin{Shaded}
\begin{Highlighting}[]
\CommentTok{# TP}
\NormalTok{boxplot.tp <-}\StringTok{ }\KeywordTok{ggplot}\NormalTok{(PeterPaul, }\KeywordTok{aes}\NormalTok{(}\DataTypeTok{x =}\NormalTok{ month, }\DataTypeTok{y =}\NormalTok{ tp_ug, }\DataTypeTok{fill =}\NormalTok{ lakename)) }\OperatorTok{+}
\StringTok{  }\KeywordTok{geom_boxplot}\NormalTok{() }\OperatorTok{+}
\StringTok{  }\KeywordTok{labs}\NormalTok{ (}\DataTypeTok{title =} \StringTok{"Total phosphorus by month"}\NormalTok{, }\DataTypeTok{fill =} \StringTok{"Lake name"}\NormalTok{,}
        \DataTypeTok{y =} \StringTok{"Total phosphorus (ug/L)"}\NormalTok{, }\DataTypeTok{x =} \StringTok{"Month"}\NormalTok{)}
  \KeywordTok{scale_fill_brewer}\NormalTok{(}\DataTypeTok{palette =} \StringTok{"Dark2"}\NormalTok{) }
\end{Highlighting}
\end{Shaded}

\begin{verbatim}
## <ggproto object: Class ScaleDiscrete, Scale, gg>
##     aesthetics: fill
##     axis_order: function
##     break_info: function
##     break_positions: function
##     breaks: waiver
##     call: call
##     clone: function
##     dimension: function
##     drop: TRUE
##     expand: waiver
##     get_breaks: function
##     get_breaks_minor: function
##     get_labels: function
##     get_limits: function
##     guide: legend
##     is_discrete: function
##     is_empty: function
##     labels: waiver
##     limits: NULL
##     make_sec_title: function
##     make_title: function
##     map: function
##     map_df: function
##     n.breaks.cache: NULL
##     na.translate: TRUE
##     na.value: NA
##     name: waiver
##     palette: function
##     palette.cache: NULL
##     position: left
##     range: <ggproto object: Class RangeDiscrete, Range, gg>
##         range: NULL
##         reset: function
##         train: function
##         super:  <ggproto object: Class RangeDiscrete, Range, gg>
##     rescale: function
##     reset: function
##     scale_name: brewer
##     train: function
##     train_df: function
##     transform: function
##     transform_df: function
##     super:  <ggproto object: Class ScaleDiscrete, Scale, gg>
\end{verbatim}

\begin{Shaded}
\begin{Highlighting}[]
\KeywordTok{print}\NormalTok{(boxplot.tp)}
\end{Highlighting}
\end{Shaded}

\includegraphics{Zarate_A05_DataVisualization_files/figure-latex/unnamed-chunk-4-2.pdf}

\begin{Shaded}
\begin{Highlighting}[]
\CommentTok{# TN}
\NormalTok{boxplot.tn <-}\StringTok{ }\KeywordTok{ggplot}\NormalTok{(PeterPaul, }\KeywordTok{aes}\NormalTok{(}\DataTypeTok{x =}\NormalTok{ month, }\DataTypeTok{y =}\NormalTok{ tn_ug, }\DataTypeTok{fill =}\NormalTok{ lakename)) }\OperatorTok{+}
\StringTok{  }\KeywordTok{geom_boxplot}\NormalTok{() }\OperatorTok{+}
\StringTok{  }\KeywordTok{labs}\NormalTok{ (}\DataTypeTok{title =} \StringTok{"Total nitrogen by month"}\NormalTok{, }\DataTypeTok{fill =} \StringTok{"Lake name"}\NormalTok{,}
        \DataTypeTok{y =} \StringTok{"Total nitrogen (ug/L)"}\NormalTok{, }\DataTypeTok{x =} \StringTok{"Month"}\NormalTok{) }
  \KeywordTok{scale_fill_brewer}\NormalTok{(}\DataTypeTok{palette =} \StringTok{"Dark2"}\NormalTok{)}
\end{Highlighting}
\end{Shaded}

\begin{verbatim}
## <ggproto object: Class ScaleDiscrete, Scale, gg>
##     aesthetics: fill
##     axis_order: function
##     break_info: function
##     break_positions: function
##     breaks: waiver
##     call: call
##     clone: function
##     dimension: function
##     drop: TRUE
##     expand: waiver
##     get_breaks: function
##     get_breaks_minor: function
##     get_labels: function
##     get_limits: function
##     guide: legend
##     is_discrete: function
##     is_empty: function
##     labels: waiver
##     limits: NULL
##     make_sec_title: function
##     make_title: function
##     map: function
##     map_df: function
##     n.breaks.cache: NULL
##     na.translate: TRUE
##     na.value: NA
##     name: waiver
##     palette: function
##     palette.cache: NULL
##     position: left
##     range: <ggproto object: Class RangeDiscrete, Range, gg>
##         range: NULL
##         reset: function
##         train: function
##         super:  <ggproto object: Class RangeDiscrete, Range, gg>
##     rescale: function
##     reset: function
##     scale_name: brewer
##     train: function
##     train_df: function
##     transform: function
##     transform_df: function
##     super:  <ggproto object: Class ScaleDiscrete, Scale, gg>
\end{verbatim}

\begin{Shaded}
\begin{Highlighting}[]
\KeywordTok{print}\NormalTok{(boxplot.tn)}
\end{Highlighting}
\end{Shaded}

\includegraphics{Zarate_A05_DataVisualization_files/figure-latex/unnamed-chunk-4-3.pdf}

\begin{Shaded}
\begin{Highlighting}[]
\CommentTok{# Combination of plots}
\NormalTok{combined <-}\StringTok{ }\KeywordTok{plot_grid}\NormalTok{(}
\NormalTok{  boxplot.temp }\OperatorTok{+}\StringTok{ }\KeywordTok{theme}\NormalTok{(}\DataTypeTok{legend.position =} \StringTok{"none"}\NormalTok{),}
\NormalTok{  boxplot.tp }\OperatorTok{+}\StringTok{ }\KeywordTok{theme}\NormalTok{(}\DataTypeTok{legend.position =} \StringTok{"none"}\NormalTok{),}
\NormalTok{  boxplot.tn }\OperatorTok{+}\StringTok{ }\KeywordTok{theme}\NormalTok{(}\DataTypeTok{legend.position =} \StringTok{"bottom"}\NormalTok{),}
  \DataTypeTok{ncol =} \DecValTok{1}\NormalTok{, }\DataTypeTok{align =} \StringTok{'hv'}\NormalTok{, }\DataTypeTok{rel_widths =} \KeywordTok{c}\NormalTok{(}\FloatTok{1.5}\NormalTok{,}\FloatTok{1.5}\NormalTok{,}\FloatTok{1.5}\NormalTok{), }\DataTypeTok{rel_heights =} \KeywordTok{c}\NormalTok{(}\FloatTok{2.5}\NormalTok{,}\DecValTok{3}\NormalTok{,}\DecValTok{4}\NormalTok{))}
\KeywordTok{print}\NormalTok{(combined)}
\end{Highlighting}
\end{Shaded}

\includegraphics{Zarate_A05_DataVisualization_files/figure-latex/unnamed-chunk-5-1.pdf}

Question: What do you observe about the variables of interest over
seasons and between lakes?

\begin{quote}
Answer: Temperatures in both lakes are similar, presenting higher
temperatures between months 7 and 8 (July and August), which correpond
to the summer months. Overall, Peter Lake has higher values of total
phosphorus and total nitrogen across the months. Regarding total
phosphorus, Paul Lake levels tends to decrease over time, while Peter
Lake levels increase. Even when all months present outliers with large
concentration of total phosphorus, these outliers are larger during the
summer months. Likewise, regarding total nitrogen, Paul Lake levels
tends to decrease over time, while Peter Lake levels increase.
\end{quote}

\begin{enumerate}
\def\labelenumi{\arabic{enumi}.}
\setcounter{enumi}{5}
\tightlist
\item
  {[}Niwot Ridge{]} Plot a subset of the litter dataset by displaying
  only the ``Needles'' functional group. Plot the dry mass of needle
  litter by date and separate by NLCD class with a color aesthetic. (no
  need to adjust the name of each land use)
\end{enumerate}

\begin{Shaded}
\begin{Highlighting}[]
\CommentTok{# Subset of litter dataset }
\NormalTok{needles.drymass <-}\StringTok{ }\KeywordTok{ggplot}\NormalTok{(}\KeywordTok{subset}\NormalTok{(Litter, functionalGroup }\OperatorTok{==}\StringTok{ "Needles"}\NormalTok{),}
                \KeywordTok{aes}\NormalTok{(}\DataTypeTok{y =}\NormalTok{ dryMass, }\DataTypeTok{x =}\NormalTok{ collectDate, }\DataTypeTok{color =}\NormalTok{ nlcdClass)) }\OperatorTok{+}
\StringTok{  }\KeywordTok{geom_point}\NormalTok{() }\OperatorTok{+}
\StringTok{  }\KeywordTok{labs}\NormalTok{(}\DataTypeTok{title =} \StringTok{"Dry mass of Needles functional group by year"}\NormalTok{,}
       \DataTypeTok{x =} \StringTok{"Date collected"}\NormalTok{, }\DataTypeTok{y =} \StringTok{"Dry mass"}\NormalTok{, }\DataTypeTok{color =} \StringTok{"Class"}\NormalTok{) }
\KeywordTok{print}\NormalTok{(needles.drymass)}
\end{Highlighting}
\end{Shaded}

\includegraphics{Zarate_A05_DataVisualization_files/figure-latex/unnamed-chunk-6-1.pdf}

\begin{enumerate}
\def\labelenumi{\arabic{enumi}.}
\setcounter{enumi}{6}
\tightlist
\item
  {[}Niwot Ridge{]} Now, plot the same plot but with NLCD classes
  separated into three facets rather than separated by color.
\end{enumerate}

\begin{Shaded}
\begin{Highlighting}[]
\NormalTok{needles <-}\StringTok{ }\KeywordTok{ggplot}\NormalTok{(}\KeywordTok{subset}\NormalTok{(Litter, functionalGroup }\OperatorTok{==}\StringTok{ "Needles"}\NormalTok{),}
                \KeywordTok{aes}\NormalTok{(}\DataTypeTok{y =}\NormalTok{ dryMass, }\DataTypeTok{x =}\NormalTok{ collectDate, }\DataTypeTok{color =}\NormalTok{ nlcdClass)) }\OperatorTok{+}
\StringTok{  }\KeywordTok{geom_point}\NormalTok{(}\DataTypeTok{size =} \DecValTok{1}\NormalTok{) }\OperatorTok{+}
\StringTok{  }\KeywordTok{labs}\NormalTok{(}\DataTypeTok{title =} \StringTok{"Dry mass of Needles functional group over time"}\NormalTok{,}
       \DataTypeTok{x =} \StringTok{"Date collected"}\NormalTok{, }\DataTypeTok{y =} \StringTok{"Dry mass"}\NormalTok{, }\DataTypeTok{color =} \StringTok{"Class"}\NormalTok{) }\OperatorTok{+}
\StringTok{  }\KeywordTok{facet_wrap}\NormalTok{(}\KeywordTok{vars}\NormalTok{(nlcdClass), }\DataTypeTok{nrow =} \DecValTok{3}\NormalTok{) }
\KeywordTok{print}\NormalTok{(needles)}
\end{Highlighting}
\end{Shaded}

\includegraphics{Zarate_A05_DataVisualization_files/figure-latex/unnamed-chunk-7-1.pdf}
Question: Which of these plots (6 vs.~7) do you think is more effective,
and why?

\begin{quote}
Answer: Plot 7 (facet) is more effective, because it allows us to
clearly see the data by class over the months, while in Plot 6 many data
points overlap with others and it's more difficult to find patterns.
\end{quote}

\end{document}
